% From https://github.com/UWIT-IAM/UWThesis

\documentclass [11pt, proquest] {uwthesis}[2015/03/03]


% fix for pandoc 1.14
\providecommand{\tightlist}{%
  \setlength{\itemsep}{0pt}\setlength{\parskip}{0pt}}

\newtheorem{theorem}{Jibberish}

%% \bibliography{references}

\hyphenation{mar-gin-al-ia}

%
% ----- apply watermark to every page
% ----- change 'stamp' to 'nostamp'
%------ to omit watermark
%
\usepackage[nostamp]{draftwatermark}
% % Use the following to make modification
\SetWatermarkText{DRAFT}
\SetWatermarkLightness{0.95}

%% for the per mil symbol
\usepackage[nointegrals]{wasysym}

%% for copyright symbol
\usepackage{textcomp}

%% to allow to rotate pages to landscape
\usepackage{lscape}
%% to adjust table column width
\usepackage{tabularx}

% suppress bottom page numbers on first page of each chapter
% because they overlap with text
\usepackage{etoolbox}
\patchcmd{\chapter}{plain}{empty}{}{}

%% for more attractive tables
\usepackage{booktabs}
\usepackage{longtable}


\usepackage{graphicx}


% Double spacing, if you want it.
% \def\dsp{\def\baselinestretch{2.0}\large\normalsize}
% \dsp

% If the Grad. Division insists that the first paragraph of a section
% be indented (like the others), then include this line:
% \usepackage{indentfirst}

%%%%%%%%%%%%%%%%%%
% If you want to use "sections" to partition your thesis
% un-comment the following:
%
% \counterwithout{section}{chapter}
% \setsecnumdepth{subsubsection}
% \def\sectionmark#1{\markboth{#1}{#1}}
% \def\subsectionmark#1{\markboth{#1}{#1}}
% \renewcommand{\thesection}{\arabic{section}}
% \renewcommand{\thesubsection}{\thesection.\arabic{subsection}}
% \makeatletter
% \let\l@subsection\l@section
% \let\l@section\l@chapter
% \makeatother
%
% \renewcommand{\thetable}{\arabic{table}}
% \renewcommand{\thefigure}{\arabic{figure}}
%
%%%%%%%%%%%%%%%%%%


%% Stuff from https://github.com/suchow/Dissertate

% The following line would print the thesis in a postscript font

% \usepackage{natbib}
% \def\bibpreamble{\protect\addcontentsline{toc}{chapter}{Bibliography}}

\setcounter{tocdepth}{1} % Print the chapter and sections to the toc
% controls depth of table of contents (toc): 0 = chapter, 1 = section, 2 = subsection

\usepackage{biblatex}

\prelimpages

%% from thesisdown
% To pass between YAML and LaTeX the dollar signs are added by CII
\Title{Diabetes and Public Housing}
\Author{Marc Macarulay}
\Year{2020}
\Program{Health Services}
\Chair{}{}{Health Services}
\Signature{}
\Signature{}
\Signature{}

% commands and environments needed by pandoc snippets
% extracted from the output of `pandoc -s`
%% Make R markdown code chunks work
\usepackage{array}
\usepackage{amssymb,amsmath}
\usepackage{ifxetex,ifluatex}
\ifxetex
  \usepackage{fontspec,xltxtra,xunicode}
  \defaultfontfeatures{Mapping=tex-text,Scale=MatchLowercase}
\else
  \ifluatex
    \usepackage{fontspec}
    \defaultfontfeatures{Mapping=tex-text,Scale=MatchLowercase}
  \else
    \usepackage[utf8]{inputenc}
  \fi
\fi
\usepackage{color}
\usepackage{fancyvrb}
\DefineShortVerb[commandchars=\\\{\}]{\|}
\DefineVerbatimEnvironment{Highlighting}{Verbatim}{commandchars=\\\{\}}
% Add ',fontsize=\small' for more characters per line
\newenvironment{Shaded}{}{}
\newcommand{\KeywordTok}[1]{\textcolor[rgb]{0.00,0.44,0.13}{\textbf{{#1}}}}
\newcommand{\DataTypeTok}[1]{\textcolor[rgb]{0.56,0.13,0.00}{{#1}}}
\newcommand{\DecValTok}[1]{\textcolor[rgb]{0.25,0.63,0.44}{{#1}}}
\newcommand{\BaseNTok}[1]{\textcolor[rgb]{0.25,0.63,0.44}{{#1}}}
\newcommand{\FloatTok}[1]{\textcolor[rgb]{0.25,0.63,0.44}{{#1}}}
\newcommand{\CharTok}[1]{\textcolor[rgb]{0.25,0.44,0.63}{{#1}}}
\newcommand{\StringTok}[1]{\textcolor[rgb]{0.25,0.44,0.63}{{#1}}}
\newcommand{\CommentTok}[1]{\textcolor[rgb]{0.38,0.63,0.69}{\textit{{#1}}}}
\newcommand{\OtherTok}[1]{\textcolor[rgb]{0.00,0.44,0.13}{{#1}}}
\newcommand{\AlertTok}[1]{\textcolor[rgb]{1.00,0.00,0.00}{\textbf{{#1}}}}
\newcommand{\FunctionTok}[1]{\textcolor[rgb]{0.02,0.16,0.49}{{#1}}}
\newcommand{\RegionMarkerTok}[1]{{#1}}
\newcommand{\ErrorTok}[1]{\textcolor[rgb]{1.00,0.00,0.00}{\textbf{{#1}}}}
\newcommand{\NormalTok}[1]{{#1}}
\newcommand{\OperatorTok}[1]{\textcolor[rgb]{0.00,0.44,0.13}{\textbf{{#1}}}}
\newcommand{\BuiltInTok}[1]{\textcolor[rgb]{0.00,0.44,0.13}{\textbf{{#1}}}}
\newcommand{\ControlFlowTok}[1]{\textcolor[rgb]{0.00,0.44,0.13}{\textbf{{#1}}}}


\ifxetex
  \usepackage[setpagesize=false, % page size defined by xetex
              unicode=false, % unicode breaks when used with xetex
              xetex,
              colorlinks=true,
              linkcolor=blue]{hyperref}
\else
  \usepackage[unicode=true,
              colorlinks=true,
              linkcolor=blue]{hyperref}
\fi
\hypersetup{breaklinks=true, pdfborder={0 0 0}}
\setlength{\parindent}{0pt}
\setlength{\parskip}{6pt plus 2pt minus 1pt}
\setlength{\emergencystretch}{3em}  % prevent overfull lines
\setcounter{secnumdepth}{2} %% controls section numbering, e.g. 1 or 1.2, or 1.2.3

\begin{document}
\titlepage
  \pagebreak


\copyrightpage

\setcounter{page}{-1}
\abstract{``Here is my abstract''}

\tableofcontents
\listoffigures
\listoftables



\textpages


\chapter{Background and Significance}\label{rmd-basics}

\section{Public Housing}\label{public-housing}

\section{Diabetes}\label{diabetes}

Diabetes is a chronic disease that is characterized by an inability of
the body to maintain a healthy blood glucose level, this can cause a
variety of symptoms that affect multiple systems in the body and can
lead to potentially life-threatening complications. The key regulator
hormone of glucose is insulin and it is produced in the pancreas. The
absence or malfunction of insulin leads to elevated blood glucose levels
called hyperglycemia. When insulin hormone is missing or ineffective the
disease is called Diabetes Mellitus, this condition has multiple types.

\subsection{Diabetes Variants}\label{diabetes-variants}

The most common diabetes variants include type I diabetes mellitus, type
II diabetes mellitus, and gestational diabetes. Type I diabetes is
usually caused by genetic factors triggering an autoimmune reaction that
results in the destruction of insulin producing cells in the pancreas.
Also known as Juvenile Diabetes, the type I classification is typically
diagnosed relatively early in life during childhood or early adulthood.
Whereas, Type II diabetes develops when the body can still produce
insulin however the amount is insuffient or when the body becomes
resistant to the effects of insulin. Type II diabetes is largely
attributed to lifestyle factors including obesity and physical activity
levels. Gestational diabetes is the least common type and occurs during
pregnancy.

Diabetes is a serious chronic disease condition without a medical cure.
Medical treatment of Diabetes is centered around exogenous insulin
replacement or use of medications that stimulate the pancreas to produce
endogenous insulin. In the absence of adequate control, diabetes can
lead to increased risk of vision loss, heart disease, stroke, kidney
failure, nerve damage, amputation of toes, feet, or legs and even
premature death; all of which have financial implications.

Many families have been left devastated by some of these complications
and are financially indebted because of hospital bills, cost of
medications, and time off work. For Type II Diabetics, a big part of
their management is lifestyle modification which includes diet control
and increased physical activity. This goal of this later method is to
promote weight loss and reduce excess fat which in turn reduces insulin
resistance and enhances disease control.(Ludwig et al., 2011)

For this reason, One avenure that public health researchers are
beginging to explore is the relationship between

several studies have examined the

Few studies have examined th eassociation between

Finding an association between publich housing and diabetes stuatus.

\section{Problem Definition}\label{problem-definition}

\chapter{Methods}\label{math-sci}

\section{Study Setting and Study
Design}\label{study-setting-and-study-design}

The current study investigates whether public housing is asociated with
risk of diabetes status among King County, WA residents who were
enrolled in Medicare and Medicaid. This study uses a descriptive
cross-sectional design. The cross-sectional design is appropriate
because it allows for an estimate of a dichotomous disease outcome at a
particular point in time ({\textbf{???}}).

The analysis of this study was conducted on a dataset compiled from the
King County \emph{Data Across Sectors for Housing and Health (DASHH)}
partnership. The findings from the original initial study have
previously been reported (Public Health - Seattle \& King County, 2018).

\section{Data Sources}\label{data-sources}

In an effort to reduce fragmented data siloes across different sectors,
the DASHH partnership was formed in 2016 between Public Health - Seattle
and King County (PHSKC), and two public housing authories, King County
Housing Authority (KCHA) and Seattle Housing Authority (SHA). The
primary objectives for DASHH were to join health and housing
administrative data together to inform and measure future interventions,
relating to policy, outreach, and program evaluation that would improve
the health of King County residents, as well as to disseminate
actionable data with key health and housing stakeholders.

The housing data provided by both KCHA and SHA originated from the US
Department of Housing and Urban Development (HUD). This data source
contained elements that included demographic information and period of
enrollment for families and individuals. Claims and enrollment for
Medicaid and Medicare data were from Washington Health Care Authority
(HCA) which was provided to PHSKC. Enrollment data contained information
on who was recieving Medicaid and Medicare benefits. Claims data
provided elements such as diagnosis codes that were used to identify
acute events and chronic conditions. All these data sources were linked
together by a unique identifier ID.

\section{Study Population}\label{study-population}

The study population were participants that were enrolled in either
Medicare or Medicaid programs. Further eligibility for study
participation included King County, Washington residency and at least 11
months of Medicare or Medicaid coverage in 2017. The overall number of
participants derived from the DASHH dataset totaled 585,372.

\subsection{Exposure Variable}\label{exposure-variable}

The exposure variable for this study was public housing assistance
status. This was extracted from the HUD-50058 form which was provided by
the PHAs. The HUD-50058 form provides information on families that
participate in public housing or Section 8 rental subsidy programs
{[}Source{]}. Housing assitance is separated into 3 main types:
\begin{itemize}
\tightlist
\item
  Housing Choice Vouchers - vouchers provided to recipients to rent
  units on the private housing market
\item
  Public housing properties and units - subisidized housing managed by
  PHAs
\item
  Project-based vouchers - subsidized housing units not managed by PHAs
\end{itemize}
Responses on the HUD-50058 form were combined into a composite public
housing binary variable. Study partipants that were not enrolled in any
of the listed housing assistance programs were coded as 0 for PHA
status. Whereas, those responses that contained any of the 3 types of
housing housing assistance was given a 1 for PHA status.

\subsection{Outcome Variable}\label{outcome-variable}

The outcome variable for this study was diabetes status.This was defined
using the Centers for Medicare and Medicaid Services (CMS) Chonic
Conditions Warehouse (CCW) algorithm {[}Source{]}. According to the CCW,
a participant meets the criteria if they have at least 1 inpatient,
skilled nursing facility, home health agency visit or 2 hospital
outpatient or carrier claims with diabetes diagnoses codes as outlined
by the chronic conditions reference list within the last 2
years{[}Source{]}. This definition does not specify diabetes variant but
instead accounts for any type diabetes diagnoses. The diabetes status
outcome variable was dichotomous, given a 0 or 1. Those that did not
meet the CCW alogrithm were coded a 0 and those that met the criteria
were coded as 1 for diabetes status.

\subsection{Potenetial Confounders}\label{potenetial-confounders}

Potential confounders were identified based on literature review. This
study considers age, race and ethnicity and gender as potential
confounding variables. Each of these variables were selected due to the
increased baseline risk for partipants to be either in public housing or
have diabetes. It is known that diabetes is an age-related disease, with
a higher risk for older populations (Selvin \& Parrinello, 2013). Age
was presented as a discrete variable for the partipants age in 2017.
Similarly, according to CDC data, racial minority groups may be
differentially at risk for both type 1 and type 2 diabetes compared to
their white counterparts (Divers et al., 2020 \& CDC (2020)). Race and
ethnicity variable was defined categorically and included: American
Indians/Alaska Natives, Asian, Asian Pacific Islander, Black/African
American, Latino, Multiple, Native Hawaiian and Pacific Islander, Other,
Unknown, and White. Gender was selected because both psychosocial and
biological factors are responsible for sex and gender diabetes risk
differences (Kautzky-Willer, Harreiter, \& Pacini, 2016). Gender was
grouped categorically and included: Female, Male, and Multiple.

\section{Analyses}\label{analyses}

As is common in epidemiological and health services research,
demographic characteristics were presented to describe the population
distribition ({\textbf{???}}). Descriptive analyses were first used to
list the percentages for each of the demographic categorical variables.
(See table 1). The demographics table is arranged by PHA status, this
included: KCHA, SHA, combined PHA and non-PHA. Although the discrete
variable for age was used in the statistical analyses, age was reported
categorically in the descriptive analyses for a simpler layout. Mean and
median age were also shown for each category.

For the statistical analyses, logistic regression models were fitted to
assess the risk of diabetes status in relation to public housing
assistance status. This analysis is appropriate for this study because
logistic reregression analyses allows for measuring the association of
an effect towards a binomial response variable by combining multiple
variables to avoid confounding ({\textbf{???}}). Two models were used to
determine the odds ratios (OR) and coresponding 95\% confidence
intervals for the association between public housing assistance and
diabetes status. The models used were the unadjusted model, without any
other variables included in the analysis and the adjusted model
including age, race and ethnicity and gender variables. In addition,
models were fit to determine the odds of diabetes for each of the public
housing authority. Similarly, the unadjusted model and the adjusted
model that included age, race and ethinicity and gender variables were
used to determine the association for the second analysis. Findings were
statistical significant if the estimates did not cross the the
confidence intervals and p-values were below \textless{}0.05 cutoff.
Analyses were conducted using R version 3.6.0.

\chapter{Results}\label{ref-labels}

Among the study participants, the proportion of people that were in the
PHA category was 10.4\% and of that, 5.9\% were with KCHA and 4.6\% with
SHA. The majority, 89.5\% did not have any public housing assitance in
2017. Descriptive analysis revealed that PHA population had a greater
proportion of people meeting the definition of diabetes at 12.7\%
compared to the non-PHA group with 9.6\%. Overall, 9.9\% were considered
to meet the definiton of diabetes and the rest, 90.1\% were not
considered to have diabetes. Additionally, the population age
distribution were different bewtween PHA status, the non-PHA category
had an older population with a median of age of 62 and a mean age of 50
compared to the PHA population with a median and mean age of 34 and 35.7
respectively.
\begin{table}

\caption{\label{tab:unnamed-chunk-1}Population Demographics}
\centering
\fontsize{12}{14}\selectfont
\begin{tabu} to \linewidth {>{\raggedright\arraybackslash}p{6.1cm}>{\raggedright}X>{\raggedright}X>{\raggedright}X>{\raggedright}X}
\toprule
Characteristics & KCHA & SHA & Combined PHA & Non-PHA\\
\midrule
 & N=34,514 (5.9\%) & N=27,044  (4.6\%) & N=60,919  (10.4\%) & N=523,814 (89.5\%)\\
\addlinespace[0.3em]
\multicolumn{5}{l}{\textbf{Age}}\\
\hspace{1em}<5 & 6.6\% & 6.1\% & 6.4\% & 5.5\%\\
\hspace{1em}5-9 & 12.0\% & 10.2\% & 11.2\% & 7.0\%\\
\hspace{1em}10-17 & 19.5\% & 14.9\% & 17.5\% & 9.8\%\\
\hspace{1em}18-29 & 12.5\% & 9.9\% & 11.3\% & 8.3\%\\
\hspace{1em}30-49 & 21.0\% & 19.3\% & 20.3\% & 11.2\%\\
\hspace{1em}50-64 & 15.3\% & 19.9\% & 17.4\% & 9.4\%\\
\hspace{1em}65-74 & 6.8\% & 11.5\% & 8.9\% & 28.0\%\\
\hspace{1em}75+ & 6.1\% & 7.9\% & 7.0\% & 20.6\%\\
\hspace{1em}Median & 39.1 years & 29.0 years & 34.0 years & 62.0 years\\
\hspace{1em}Mean & 33.3 years & 38.7 years & 35.7 years & 50.0 years\\
\addlinespace[0.3em]
\multicolumn{5}{l}{\textbf{Race and Ethnicity}}\\
\hspace{1em}American Indian or Alaska Native & 0.8\% & 1.4\% & 1.0\% & 0.8\%\\
\hspace{1em}Asian & 5.5\% & 11.7\% & 8.3\% & 6.9\%\\
\hspace{1em}Asian Pacific Islander & 0.1\% & 0.2\% & 0.2\% & 3.5\%\\
\hspace{1em}Black/African American & 36.9\% & 44.9\% & 40.2\% & 7.9\%\\
\hspace{1em}Latino & 3.8\% & 2.8\% & 3.4\% & 6.5\%\\
\hspace{1em}Multiple & 15.5\% & 10.2\% & 13.2\% & 8.0\%\\
\hspace{1em}Native Hawaiian or Pacific Islander & 2.3\% & 1.9\% & 2.1\% & 2.4\%\\
\hspace{1em}Other & 0.0\% & 0.0\% & 0.0\% & 0.8\%\\
\hspace{1em}White & 30.1\% & 22.3\% & 26.8\% & 56.1\%\\
\hspace{1em}Unknown & 5.0\% & 4.5\% & 4.8\% & 6.9\%\\
\addlinespace[0.3em]
\multicolumn{5}{l}{\textbf{Gender}}\\
\hspace{1em}Female & 58.6\% & 53.5\% & 56.3\% & 52.4\%\\
\hspace{1em}Male & 40.6\% & 45.7\% & 42.9\% & 47.2\%\\
\hspace{1em}Multiple & 0.8\% & 0.8\% & 0.8\% & 0.4\%\\
\bottomrule
\multicolumn{5}{l}{\textit{Note: }}\\
\multicolumn{5}{l}{Percentages may not add up to 100 because of missing data}\\
\end{tabu}
\end{table}
For the primary analysis, the assocation between diabetes status and
public housing assistance, the crude model showed that the odds ratio of
having diabetes was 1.34 fold greater for those with public housing
assistance (table 2). In the adjusted model, PHA residents were 94\%
more likely to meet the definition of diabetes compared to those that
were non-PHA residents.
\begin{longtable}[]{@{}lll@{}}
\caption{\label{tab:inher} Association between PHA Status and
Diabetes}\tabularnewline
\toprule
Housing Status & Model 1 & Model 2\tabularnewline
\midrule
\endfirsthead
\toprule
Housing Status & Model 1 & Model 2\tabularnewline
\midrule
\endhead
Non-PHA & Referent & Referent\tabularnewline
PHA & 1.34 (95\% CI:1.31-1.38) & 1.94 (95\% CI:1.88-1.99)\tabularnewline
\bottomrule
\end{longtable}
In the second analysis, the association between diabetes status and the
specific public housing authorities, the crude model showed that the
odds of meeting the definiton of diabetes were 1.28 times greater among
KCHA residents and 1.42 times greater among SHA residents. The adjusted
model revealed that among KCHA residents the odds of meeting the
definition of diabetes were 2.16 times higher and 1.70 for SHA residents
compared to non-PHA residents.
\begin{longtable}[]{@{}lll@{}}
\caption{\label{tab:inher} Association between the Public Housing
Authorities and Diabetes}\tabularnewline
\toprule
Status & Model 1 & Model 2\tabularnewline
\midrule
\endfirsthead
\toprule
Status & Model 1 & Model 2\tabularnewline
\midrule
\endhead
Non-PHA & Referent & Referent\tabularnewline
KCHA & 1.28 (CI: 1.24-1.33) & 2.16 (CI: 2.09-2.25)\tabularnewline
SHA & 1.42 (CI: 1.38-1.48) & 1.70 (CI: 1.64-1.77)\tabularnewline
\bottomrule
\end{longtable}
\chapter*{Conclusion}\label{conclusion}
\addcontentsline{toc}{chapter}{Conclusion}

If we don't want Conclusion to have a chapter number next to it, we can
add the \texttt{\{-\}} attribute.

\textbf{More info}

And here's some other random info: the first paragraph after a chapter
title or section head \emph{shouldn't be} indented, because indents are
to tell the reader that you're starting a new paragraph. Since that's
obvious after a chapter or section title, proper typesetting doesn't add
an indent there.

\appendix

\chapter{Appendix}\label{appendix}

\chapter*{Colophon}\label{colophon}
\addcontentsline{toc}{chapter}{Colophon}

This document is set in \href{https://github.com/georgd/EB-Garamond}{EB
Garamond}, \href{https://github.com/adobe-fonts/source-code-pro/}{Source
Code Pro} and \href{http://www.latofonts.com/lato-free-fonts/}{Lato}.
The body text is set at 11pt with \(\familydefault\).

It was written in R Markdown and \(\LaTeX\), and rendered into PDF using
\href{https://github.com/benmarwick/huskydown}{huskydown} and
\href{https://github.com/rstudio/bookdown}{bookdown}.

This document was typeset using the XeTeX typesetting system, and the
\href{http://staff.washington.edu/fox/tex/}{University of Washington
Thesis class} class created by Jim Fox. Under the hood, the
\href{https://github.com/UWIT-IAM/UWThesis}{University of Washington
Thesis LaTeX template} is used to ensure that documents conform
precisely to submission standards. Other elements of the document
formatting source code have been taken from the
\href{https://github.com/stevenpollack/ucbthesis}{Latex, Knitr, and
RMarkdown templates for UC Berkeley's graduate thesis}, and
\href{https://github.com/suchow/Dissertate}{Dissertate: a LaTeX
dissertation template to support the production and typesetting of a PhD
dissertation at Harvard, Princeton, and NYU}

The source files for this thesis, along with all the data files, have
been organised into an R package, xxx, which is available at
\url{https://github.com/xxx/xxx}. A hard copy of the thesis can be found
in the University of Washington library.

This version of the thesis was generated on 2020-05-23 21:07:20. The
repository is currently at this commit:

The computational environment that was used to generate this version is
as follows:
\begin{verbatim}
- Session info ---------------------------------------------------------------
 setting  value                       
 version  R version 3.6.1 (2019-07-05)
 os       Windows 10 x64              
 system   x86_64, mingw32             
 ui       RTerm                       
 language (EN)                        
 collate  English_United States.1252  
 ctype    English_United States.1252  
 tz       America/Los_Angeles         
 date     2020-05-23                  

- Packages -------------------------------------------------------------------
 package     * version date       lib source                               
 assertthat    0.2.1   2019-03-21 [1] CRAN (R 3.6.1)                       
 backports     1.1.4   2019-04-10 [1] CRAN (R 3.6.0)                       
 bookdown      0.18.1  2020-05-01 [1] Github (rstudio/bookdown@cd97d40)    
 callr         3.3.1   2019-07-18 [1] CRAN (R 3.6.1)                       
 cli           1.1.0   2019-03-19 [1] CRAN (R 3.6.1)                       
 colorspace    1.4-1   2019-03-18 [1] CRAN (R 3.6.1)                       
 crayon        1.3.4   2017-09-16 [1] CRAN (R 3.6.1)                       
 desc          1.2.0   2018-05-01 [1] CRAN (R 3.6.2)                       
 devtools    * 2.2.1   2019-09-24 [1] CRAN (R 3.6.2)                       
 digest        0.6.20  2019-07-04 [1] CRAN (R 3.6.1)                       
 dplyr       * 0.8.3   2019-07-04 [1] CRAN (R 3.6.1)                       
 ellipsis      0.3.0   2019-09-20 [1] CRAN (R 3.6.2)                       
 evaluate      0.14    2019-05-28 [1] CRAN (R 3.6.1)                       
 fs            1.3.1   2019-05-06 [1] CRAN (R 3.6.1)                       
 ggplot2       3.2.0   2019-06-16 [1] CRAN (R 3.6.0)                       
 git2r         0.26.1  2019-06-29 [1] CRAN (R 3.6.2)                       
 glue          1.3.1   2019-03-12 [1] CRAN (R 3.6.1)                       
 gtable        0.3.0   2019-03-25 [1] CRAN (R 3.6.1)                       
 hms           0.5.0   2019-07-09 [1] CRAN (R 3.6.1)                       
 htmltools     0.4.0   2019-10-04 [1] CRAN (R 3.6.2)                       
 httr          1.4.0   2018-12-11 [1] CRAN (R 3.6.1)                       
 huskydown   * 0.0.5   2020-05-01 [1] Github (benmarwick/huskydown@a909835)
 kableExtra  * 1.1.0   2019-03-16 [1] CRAN (R 3.6.3)                       
 knitr       * 1.27    2020-01-16 [1] CRAN (R 3.6.2)                       
 lazyeval      0.2.2   2019-03-15 [1] CRAN (R 3.6.1)                       
 magrittr    * 1.5     2014-11-22 [1] CRAN (R 3.6.1)                       
 memoise       1.1.0   2017-04-21 [1] CRAN (R 3.6.2)                       
 munsell       0.5.0   2018-06-12 [1] CRAN (R 3.6.1)                       
 pillar        1.4.2   2019-06-29 [1] CRAN (R 3.6.1)                       
 pkgbuild      1.0.6   2019-10-09 [1] CRAN (R 3.6.2)                       
 pkgconfig     2.0.2   2018-08-16 [1] CRAN (R 3.6.1)                       
 pkgload       1.0.2   2018-10-29 [1] CRAN (R 3.6.2)                       
 prettyunits   1.0.2   2015-07-13 [1] CRAN (R 3.6.1)                       
 processx      3.4.0   2019-07-03 [1] CRAN (R 3.6.1)                       
 ps            1.3.0   2018-12-21 [1] CRAN (R 3.6.1)                       
 purrr         0.3.3   2019-10-18 [1] CRAN (R 3.6.2)                       
 R6            2.4.0   2019-02-14 [1] CRAN (R 3.6.1)                       
 Rcpp          1.0.1   2019-03-17 [1] CRAN (R 3.6.1)                       
 readr         1.3.1   2018-12-21 [1] CRAN (R 3.6.3)                       
 remotes       2.1.0   2019-06-24 [1] CRAN (R 3.6.2)                       
 rlang         0.4.3   2020-01-24 [1] CRAN (R 3.6.2)                       
 rmarkdown     2.1     2020-01-20 [1] CRAN (R 3.6.3)                       
 rprojroot     1.3-2   2018-01-03 [1] CRAN (R 3.6.1)                       
 rstudioapi    0.10    2019-03-19 [1] CRAN (R 3.6.1)                       
 rvest         0.3.4   2019-05-15 [1] CRAN (R 3.6.1)                       
 scales        1.0.0   2018-08-09 [1] CRAN (R 3.6.1)                       
 sessioninfo   1.1.1   2018-11-05 [1] CRAN (R 3.6.2)                       
 stringi       1.4.3   2019-03-12 [1] CRAN (R 3.6.0)                       
 stringr       1.4.0   2019-02-10 [1] CRAN (R 3.6.1)                       
 testthat      2.3.1   2019-12-01 [1] CRAN (R 3.6.2)                       
 tibble        2.1.3   2019-06-06 [1] CRAN (R 3.6.1)                       
 tidyselect    0.2.5   2018-10-11 [1] CRAN (R 3.6.1)                       
 usethis     * 1.6.1   2020-04-29 [1] CRAN (R 3.6.3)                       
 vctrs         0.2.0   2019-07-05 [1] CRAN (R 3.6.1)                       
 viridisLite   0.3.0   2018-02-01 [1] CRAN (R 3.6.1)                       
 webshot       0.5.2   2019-11-22 [1] CRAN (R 3.6.2)                       
 withr         2.1.2   2018-03-15 [1] CRAN (R 3.6.1)                       
 xfun          0.8     2019-06-25 [1] CRAN (R 3.6.1)                       
 xml2          1.2.0   2018-01-24 [1] CRAN (R 3.6.1)                       
 yaml          2.2.0   2018-07-25 [1] CRAN (R 3.6.0)                       
 zeallot       0.1.0   2018-01-28 [1] CRAN (R 3.6.1)                       

[1] C:/Users/Marc/Documents/R/win-library/3.6
[2] C:/Program Files/R/R-3.6.1/library
\end{verbatim}
\backmatter

\chapter*{References}\label{references}
\addcontentsline{toc}{chapter}{References}

\markboth{References}{References}

\noindent

\setlength{\parindent}{-0.20in} \setlength{\leftskip}{0.20in}
\setlength{\parskip}{8pt}

\hypertarget{refs}{}
\hypertarget{ref-CDC2020}{}
CDC. (2020). \emph{National Diabetes Statistics Report 2020. Estimates
of diabetes and its burden in the United States.}

\hypertarget{ref-Divers2020}{}
Divers, J., Mayer-Davis, E. J., Lawrence, J. M., Isom, S., Dabelea, D.,
Dolan, L., \ldots{} Wagenknecht, L. E. (2020). Trends in Incidence of
Type 1 and Type 2 Diabetes Among Youths --- Selected Counties and Indian
Reservations, United States, 2002--2015. \emph{MMWR. Morbidity and
Mortality Weekly Report}, \emph{69}(6), 161--165.
\url{http://doi.org/10.15585/mmwr.mm6906a3}

\hypertarget{ref-Kautzky-Willer2016}{}
Kautzky-Willer, A., Harreiter, J., \& Pacini, G. (2016, June). Sex and
gender differences in risk, pathophysiology and complications of type 2
diabetes mellitus. Endocrine Society.
\url{http://doi.org/10.1210/er.2015-1137}

\hypertarget{ref-Ludwig2011}{}
Ludwig, J., Sanbonmatsu, L., Gennetian, L., Adam, E., Duncan, G. J.,
Katz, L. F., \ldots{} McDade, T. W. (2011). Neighborhoods, obesity, and
diabetes - A randomized social experiment. \emph{New England Journal of
Medicine}, \emph{365}(16), 1509--1519.
\url{http://doi.org/10.1056/NEJMsa1103216}

\hypertarget{ref-PublicHealth-SeattleandKingCounty2018}{}
Public Health - Seattle \& King County. (2018). \emph{King County Data
Across Sectors for Housing and Health, 2018}. Public Health - Seattle \&
King County.

\hypertarget{ref-Selvin2013}{}
Selvin, E., \& Parrinello, C. M. (2013). Age-related differences in
glycaemic control in diabetes. NIH Public Access.
\url{http://doi.org/10.1007/s00125-013-3078-7}
\end{document}
