% From https://github.com/UWIT-IAM/UWThesis

\documentclass [11pt, proquest] {uwthesis}[2015/03/03]


% fix for pandoc 1.14
\providecommand{\tightlist}{%
  \setlength{\itemsep}{0pt}\setlength{\parskip}{0pt}}

\newtheorem{theorem}{Jibberish}

%% \bibliography{references}

\hyphenation{mar-gin-al-ia}

%
% ----- apply watermark to every page
% ----- change 'stamp' to 'nostamp'
%------ to omit watermark
%
\usepackage[nostamp]{draftwatermark}
% % Use the following to make modification
\SetWatermarkText{DRAFT}
\SetWatermarkLightness{0.95}

%% for the per mil symbol
\usepackage[nointegrals]{wasysym}

%% for copyright symbol
\usepackage{textcomp}

%% to allow to rotate pages to landscape
\usepackage{lscape}
%% to adjust table column width
\usepackage{tabularx}

% suppress bottom page numbers on first page of each chapter
% because they overlap with text
\usepackage{etoolbox}
\patchcmd{\chapter}{plain}{empty}{}{}

%% for more attractive tables
\usepackage{booktabs}
\usepackage{longtable}


\usepackage{graphicx}


% Double spacing, if you want it.
% \def\dsp{\def\baselinestretch{2.0}\large\normalsize}
% \dsp

% If the Grad. Division insists that the first paragraph of a section
% be indented (like the others), then include this line:
% \usepackage{indentfirst}

%%%%%%%%%%%%%%%%%%
% If you want to use "sections" to partition your thesis
% un-comment the following:
%
% \counterwithout{section}{chapter}
% \setsecnumdepth{subsubsection}
% \def\sectionmark#1{\markboth{#1}{#1}}
% \def\subsectionmark#1{\markboth{#1}{#1}}
% \renewcommand{\thesection}{\arabic{section}}
% \renewcommand{\thesubsection}{\thesection.\arabic{subsection}}
% \makeatletter
% \let\l@subsection\l@section
% \let\l@section\l@chapter
% \makeatother
%
% \renewcommand{\thetable}{\arabic{table}}
% \renewcommand{\thefigure}{\arabic{figure}}
%
%%%%%%%%%%%%%%%%%%


%% Stuff from https://github.com/suchow/Dissertate

% The following line would print the thesis in a postscript font

% \usepackage{natbib}
% \def\bibpreamble{\protect\addcontentsline{toc}{chapter}{Bibliography}}

\setcounter{tocdepth}{1} % Print the chapter and sections to the toc
% controls depth of table of contents (toc): 0 = chapter, 1 = section, 2 = subsection

\usepackage{biblatex}

\prelimpages

%% from thesisdown
% To pass between YAML and LaTeX the dollar signs are added by CII
\Title{Diabetes and Public Housing}
\Author{Marc Macarulay}
\Year{2020}
\Program{Health Services}
\Chair{}{}{Health Services}
\Signature{}
\Signature{}
\Signature{}

% commands and environments needed by pandoc snippets
% extracted from the output of `pandoc -s`
%% Make R markdown code chunks work
\usepackage{array}
\usepackage{amssymb,amsmath}
\usepackage{ifxetex,ifluatex}
\ifxetex
  \usepackage{fontspec,xltxtra,xunicode}
  \defaultfontfeatures{Mapping=tex-text,Scale=MatchLowercase}
\else
  \ifluatex
    \usepackage{fontspec}
    \defaultfontfeatures{Mapping=tex-text,Scale=MatchLowercase}
  \else
    \usepackage[utf8]{inputenc}
  \fi
\fi
\usepackage{color}
\usepackage{fancyvrb}
\DefineShortVerb[commandchars=\\\{\}]{\|}
\DefineVerbatimEnvironment{Highlighting}{Verbatim}{commandchars=\\\{\}}
% Add ',fontsize=\small' for more characters per line
\newenvironment{Shaded}{}{}
\newcommand{\KeywordTok}[1]{\textcolor[rgb]{0.00,0.44,0.13}{\textbf{{#1}}}}
\newcommand{\DataTypeTok}[1]{\textcolor[rgb]{0.56,0.13,0.00}{{#1}}}
\newcommand{\DecValTok}[1]{\textcolor[rgb]{0.25,0.63,0.44}{{#1}}}
\newcommand{\BaseNTok}[1]{\textcolor[rgb]{0.25,0.63,0.44}{{#1}}}
\newcommand{\FloatTok}[1]{\textcolor[rgb]{0.25,0.63,0.44}{{#1}}}
\newcommand{\CharTok}[1]{\textcolor[rgb]{0.25,0.44,0.63}{{#1}}}
\newcommand{\StringTok}[1]{\textcolor[rgb]{0.25,0.44,0.63}{{#1}}}
\newcommand{\CommentTok}[1]{\textcolor[rgb]{0.38,0.63,0.69}{\textit{{#1}}}}
\newcommand{\OtherTok}[1]{\textcolor[rgb]{0.00,0.44,0.13}{{#1}}}
\newcommand{\AlertTok}[1]{\textcolor[rgb]{1.00,0.00,0.00}{\textbf{{#1}}}}
\newcommand{\FunctionTok}[1]{\textcolor[rgb]{0.02,0.16,0.49}{{#1}}}
\newcommand{\RegionMarkerTok}[1]{{#1}}
\newcommand{\ErrorTok}[1]{\textcolor[rgb]{1.00,0.00,0.00}{\textbf{{#1}}}}
\newcommand{\NormalTok}[1]{{#1}}
\newcommand{\OperatorTok}[1]{\textcolor[rgb]{0.00,0.44,0.13}{\textbf{{#1}}}}
\newcommand{\BuiltInTok}[1]{\textcolor[rgb]{0.00,0.44,0.13}{\textbf{{#1}}}}
\newcommand{\ControlFlowTok}[1]{\textcolor[rgb]{0.00,0.44,0.13}{\textbf{{#1}}}}


\ifxetex
  \usepackage[setpagesize=false, % page size defined by xetex
              unicode=false, % unicode breaks when used with xetex
              xetex,
              colorlinks=true,
              linkcolor=blue]{hyperref}
\else
  \usepackage[unicode=true,
              colorlinks=true,
              linkcolor=blue]{hyperref}
\fi
\hypersetup{breaklinks=true, pdfborder={0 0 0}}
\setlength{\parindent}{0pt}
\setlength{\parskip}{6pt plus 2pt minus 1pt}
\setlength{\emergencystretch}{3em}  % prevent overfull lines
\setcounter{secnumdepth}{2} %% controls section numbering, e.g. 1 or 1.2, or 1.2.3

\begin{document}
\titlepage
  \pagebreak


\copyrightpage

\setcounter{page}{-1}
\abstract{Public housing residents have worse health than the general population
including chronic diseases such as diabetes. Estimating diabetes status
for public housing residents are essential for public health agencies to
implement efficient health promotion programs within populations that
have a greater need. This cross-sectional design study was conducted
using data collected from Data Across Sectors for Housing and Health
partnership in King County, WA. Total population included were Medicaid
and Medicare beneficiaries compiled from Washington Health Care
Authority (n= 585,372). Associations between public housing and diabetes
status were estimated using odds ratios and corresponding 95\%
confidence intervals from crude and adjusted models. Further
associations were determined between individual public housing
authorities and diabetes status using the same crude and adjusted
models. Among study participants 10.4\% received some form public
housing assistance in 2017. In total, 9.9\% were considered to have
diabetes and 90.1\% were not considered to have diabetes. The adjusted
model revealed that public housing residents were 1.94 times (95\% CI:
1.88-1.99) more likely to meet the diabetes definition compared to those
not receiving housing assitance. These findings suggest that public
housing residents are more likely to be diabetic. Further studies should
explore the relationship between public housing and diabetes over a
longer period to discern the direction of the association over time.
Findings from this study can be used to inform future interventions for
diabetes treatment by both public health and housing agencies.}

\tableofcontents
\listoffigures
\listoftables



\textpages


\chapter{Background and Significance}\label{rmd-basics}

\section{Public Housing}\label{public-housing}

Housing is widely acknowledged as an important social determinant of
health (Thomson, Thomas, Sellstrom, \& Petticrew, 2013). Health outcomes
driven by housing are mediated by housing quality, safety, stability,
and affordability (Taylor, 2018). There are well established links
between housing quality and morbidity ranging from mental disorders,
injuries, infectious diseases, and chronic diseases (Krieger \& Higgins,
2002). There is a growing body of evidence associating substandard
housing with poor health outcomes, but the relationship between public
housing and health is minimally explored. Public housing provides decent
and safe subsidized rental housing for eligible populations including
low-income families, the elderly, and persons with disabilities (HUD,
2020). However, relevant studies have shown that public housing
residents have worse health outcomes than other city residents
(Digenis-Bury, Brooks, Chen, Ostrem, \& Horsburgh, 2008 ; Manjarrez,
Popkin, \& Guernsey, 2007). Even less understood is how public housing
assistance impacts chronic health conditions like diabetes.

\section{Diabetes}\label{diabetes}

Diabetes is a chronic disease that is characterized by an inability of
the body to maintain a healthy blood glucose level, this can cause a
variety of symptoms that affect multiple systems in the body and can
lead to potentially life-threatening complications. The key regulator
hormone of glucose is insulin and it is produced in the pancreas. The
absence or malfunction of insulin leads to elevated blood glucose levels
called hyperglycemia. When insulin hormone is missing or ineffective the
disease is called Diabetes Mellitus and this condition has multiple
types.

\subsection{Diabetes Variants}\label{diabetes-variants}

The most common diabetes variants include type I diabetes mellitus, type
II diabetes mellitus, and gestational diabetes. Type I diabetes is
usually caused by genetic factors triggering an autoimmune reaction that
results in the destruction of insulin producing cells in the pancreas.
Also known as Juvenile Diabetes, the type I classification is typically
diagnosed relatively early in life during childhood or early adulthood.
Whereas Type II diabetes develops when the body can still produce
insulin however the amount is insufficient or when the body becomes
resistant to the effects of insulin. Type II diabetes is largely
attributed to lifestyle factors. Gestational diabetes is the least
common type and occurs during pregnancy. The prevalence of type II
diabetes are much higher than type I. In the US, type II and type I
diabetes account for approximately 91\% and 6\% of all diagnosed
diabetes cases (Bullard et al., 2018).

Diabetes is a serious chronic condition without a medical cure. The
treatment for diabetes involves disease prevention and management.
Medical treatment of diabetes primarily consists of exogenous insulin
replacement or use of medications that stimulate the pancreas to produce
endogenous insulin. Without adequate blood control, diabetes can lead to
increased risk of other conditions including vision loss, heart disease,
stroke, kidney failure, nerve damage, amputation, and even premature
death.

\section{Problem Definition}\label{problem-definition}

Disease management for type II diabetics focuses on lifestyle
modification such as diet control and increased physical activity. The
goal is to promote weight loss and reduce excess fat that subsequently
reduces insulin resistance and enhances disease control. However, other
determinants of health have been recognized to impact the effectiveness
of diabetes management, namely healthcare access, cultural and social
support, economic stability and built environments (Clark, 2014).
Housing instability and food insecurity in particular have been shown to
reduce diabetes management self-efficacy in low income adults
(Vijayaraghavan, Jacobs, Seligman, \& Fernandez, 2011).

Again, while there are numerous published literature on the association
between substandard housing and health outcomes, few studies
specifically examine the relationship between public housing and
diabetes. For this reason, the current study aimed to explore this
public health issue within a local context in King County, WA.

In the effort to decrease the gap of knowledge between the junction of
public housing and health, Public Health Seattle and King County (PHSKC)
formed a unique partnership with King County Housing Authority (KCHA),
Seattle Housing Authority (SHA) enabling data to be shared across
sectors with the intention of informing and measuring future
interventions that would improve the health of the county residents.
This research aims to use the provided data to contribute to the
literature on the association between public housing and diabetes among
Medicaid and Medicare patients. The findings of the study could help
identify where resources for diabetes prevention and management might be
more effective.

\chapter{Methods}\label{methods}

\section{Study Setting and Study
Design}\label{study-setting-and-study-design}

The current study investigates whether public housing is associated with
risk of diabetes status among King County, WA residents who were
enrolled in Medicare and Medicaid. This study uses a descriptive
cross-sectional design. The cross-sectional design is appropriate
because it allows for an estimate of a dichotomous disease outcome at a
particular point in time (Greenland \& Morgenstern, 1988).

The analysis of this study was conducted on a dataset compiled from the
King County \emph{Data Across Sectors for Housing and Health (DASHH)}
partnership. The findings from the original initial study have
previously been reported (Public Health - Seattle \& King County, 2018).

\section{Data Sources}\label{data-sources}

In an effort to reduce fragmented data siloes across different sectors,
the DASHH partnership was formed in 2016 between Public Health - Seattle
and King County , and two public housing authories, King County Housing
Authority and Seattle Housing Authority. The primary objectives for
DASHH were to join health and housing administrative data together to
inform and measure future interventions, relating to policy, outreach,
and program evaluation that would improve the health of King County
residents, as well as to disseminate actionable data with key health and
housing stakeholders.

The housing data provided by both KCHA and SHA originated from the US
Department of Housing and Urban Development (HUD). This data source
contained elements that included demographic information and period of
enrollment for families and individuals. Claims and enrollment for
Medicaid and Medicare data were from Washington Health Care Authority
(HCA) which was provided to PHSKC. Enrollment data contained information
on who was receiving Medicaid and Medicare benefits. Claims data
provided elements such as diagnosis codes that were used to identify
acute events and chronic conditions. All these data sources were linked
together by a unique identifier ID.

\section{Study Population}\label{study-population}

The study population were participants that were enrolled in either
Medicare or Medicaid programs. Further eligibility for study
participation included King County, Washington residency and at least 11
months of Medicare or Medicaid coverage in 2017. The minimum coverage
restriction provides a more accurate representation of participants with
full Medicaid and Medicare insurance benefits. The overall number of
participants derived from the DASHH dataset totaled 585,372. \#\#\#
Exposure Variable The exposure variable for this study was public
housing assistance status. This was extracted from the HUD-50058 form
which was provided by the PHAs. The HUD-50058 form provides information
on families that participate in public housing or Section 8 rental
subsidy programs (HUD, 2020). Housing assistance is separated into 3
main types:
\begin{itemize}
\tightlist
\item
  Housing Choice Vouchers - vouchers provided to recipients to rent
  units on the private housing market
\item
  Public housing properties and units - subsidized housing managed by
  PHAs
\item
  Project-based vouchers - subsidized housing units not managed by PHAs
\end{itemize}
Responses on the HUD-50058 form were combined into a composite public
housing binary variable. Study participants that were not enrolled in
any of the listed housing assistance programs were coded as 0 for PHA
status. Whereas those responses that contained any of the 3 types of
housing assistance was given a 1 for PHA status.

\subsection{Outcome Variable}\label{outcome-variable}

The outcome variable for this study was diabetes status. This was
defined using the Centers for Medicare and Medicaid Services (CMS)
Chronic Conditions Warehouse (CCW) algorithm (Centers for Medicare and
Medicaid Services, 2020). According to the CCW, a participant meets the
criteria if they have at least 1 inpatient, skilled nursing facility,
home health agency visit or 2 hospital outpatient or carrier claims with
diabetes diagnoses codes as outlined by the chronic conditions reference
list within the last 2 years(Centers for Medicare and Medicaid Services,
2020). This definition does not specify diabetes variant but instead
accounts for any type diabetes diagnoses. The diabetes status outcome
variable was dichotomous, given a 0 or 1. Those that did not meet the
CCW algorithm were coded a 0 and those that met the criteria were coded
as 1 for diabetes status.

\subsection{Potential Confounders}\label{potential-confounders}

Potential confounders were identified based on literature review. This
study considers age, race and ethnicity and gender as potential
confounding variables. Each of these variables were selected due to the
increased baseline risk for participants to be either in public housing
or have diabetes. It is known that diabetes is an age-related disease,
with a higher risk for older populations (Selvin \& Parrinello, 2013).
Age was presented as a discrete variable for the participants age in
2017. Similarly, according to CDC data, racial minority groups may be
differentially at risk for both type 1 and type 2 diabetes compared to
their white counterparts (CDC, 2020; Divers et al., 2020 ). Race and
ethnicity variable was defined categorically and included: American
Indians/Alaska Natives, Asian, Asian Pacific Islander, Black/African
American, Latino, Multiple, Native Hawaiian and Pacific Islander, Other,
Unknown, and White. Gender was selected because both psychosocial and
biological factors are responsible for sex and gender diabetes risk
differences (Kautzky-Willer, Harreiter, \& Pacini, 2016). Gender was
grouped categorically and included: Female, Male, and Multiple.

\section{Analyses}\label{analyses}

As is common in epidemiological and health services research,
demographic characteristics were presented to describe the population
distribution (Hayes-Larson, Kezios, Mooney, \& Lovasi, 2019).
Descriptive analyses were first used to list the percentages for each of
the demographic categorical variables. (See table 1). The demographics
table is arranged by PHA status, this included: KCHA, SHA, combined PHA
and non-PHA. Although the discrete variable for age was used in the
statistical analyses, age was reported categorically in the descriptive
analyses for a simpler layout. Mean and median age were also shown for
each category.

For the statistical analyses, logistic regression models were fitted to
assess the risk of diabetes status in relation to public housing
assistance status. This analysis is well-suited for this study because
logistic regression analyses allows for measuring the association of an
effect towards a binomial response variable by combining multiple
variables to avoid confounding (Sperandei, 2014). Given the binary
outcome variable of diabetes status, logistic regression is an
appropriate choice.

There were two main statistical analyses were performed in this study,
the relationship between public housing and diabetes as well as the
relationship between the specific public housing authorities and
diabetes. Two models were used to determine the odds ratios (OR) and
corresponding 95\% confidence intervals for the association between
public housing assistance and diabetes status. The models used were the
unadjusted model, without any other variables included in the analysis
and the adjusted model including age, race and ethnicity and gender
variables. In addition, these models were also fit to determine the odds
ratio of diabetes status in relation to public housing authority.
Similarly, the unadjusted model and the adjusted model that included
age, race and ethnicity and gender variables were used to determine the
association for the second analysis. Findings were statistically
significant if the estimates did not cross the confidence intervals and
p-values were below \textless{}0.05 threshold. Analyses were conducted
using R version 3.6.0.

\chapter{Results}\label{ref-labels}

\section{Descriptive Statistics}\label{descriptive-statistics}

Among the study participants, the proportion of people that were in the
PHA category was 10.4\% and of that, 5.9\% were with KCHA and 4.6\% with
SHA (See Table \ref{tab:table1}). The majority of the study
participants, 89.5\% did not recieve any type of public housing
assistance in 2017. Descriptive analysis revealed that PHA population
had a greater proportion of people meeting the definition of diabetes at
12.7\% compared to the non-PHA group with 9.6\%. Overall, 9.9\% were
considered to meet the definition of diabetes and the rest, 90.1\% were
not considered to have diabetes.

Additionally, the population age distribution were different between PHA
status, the non-PHA category had an older population with a median of
age of 62 and a mean age of 50 compared to the PHA population with a
median and mean age of 34 and 35.7 respectively. The PHA group were more
racial and ethnically diverse than the non-PHA group. However, the
gender distribution between the two groups were similar.
\begin{table}

\caption{\label{tab:table1}Population Demographics}
\centering
\fontsize{12}{14}\selectfont
\begin{tabu} to \linewidth {>{\raggedright\arraybackslash}p{6.1cm}>{\raggedright}X>{\raggedright}X>{\raggedright}X>{\raggedright}X}
\toprule
Characteristics & KCHA & SHA & Combined PHA & Non-PHA\\
\midrule
 & N=34,514 (5.9\%) & N=27,044  (4.6\%) & N=60,919  (10.4\%) & N=523,814 (89.5\%)\\
\addlinespace[0.3em]
\multicolumn{5}{l}{\textbf{Age}}\\
\hspace{1em}<5 & 6.6\% & 6.1\% & 6.4\% & 5.5\%\\
\hspace{1em}5-9 & 12.0\% & 10.2\% & 11.2\% & 7.0\%\\
\hspace{1em}10-17 & 19.5\% & 14.9\% & 17.5\% & 9.8\%\\
\hspace{1em}18-29 & 12.5\% & 9.9\% & 11.3\% & 8.3\%\\
\hspace{1em}30-49 & 21.0\% & 19.3\% & 20.3\% & 11.2\%\\
\hspace{1em}50-64 & 15.3\% & 19.9\% & 17.4\% & 9.4\%\\
\hspace{1em}65-74 & 6.8\% & 11.5\% & 8.9\% & 28.0\%\\
\hspace{1em}75+ & 6.1\% & 7.9\% & 7.0\% & 20.6\%\\
\hspace{1em}Median & 39.1 years & 29.0 years & 34.0 years & 62.0 years\\
\hspace{1em}Mean & 33.3 years & 38.7 years & 35.7 years & 50.0 years\\
\addlinespace[0.3em]
\multicolumn{5}{l}{\textbf{Race and Ethnicity}}\\
\hspace{1em}American Indian or Alaska Native & 0.8\% & 1.4\% & 1.0\% & 0.8\%\\
\hspace{1em}Asian & 5.5\% & 11.7\% & 8.3\% & 6.9\%\\
\hspace{1em}Asian Pacific Islander & 0.1\% & 0.2\% & 0.2\% & 3.5\%\\
\hspace{1em}Black/African American & 36.9\% & 44.9\% & 40.2\% & 7.9\%\\
\hspace{1em}Latino & 3.8\% & 2.8\% & 3.4\% & 6.5\%\\
\hspace{1em}Multiple & 15.5\% & 10.2\% & 13.2\% & 8.0\%\\
\hspace{1em}Native Hawaiian or Pacific Islander & 2.3\% & 1.9\% & 2.1\% & 2.4\%\\
\hspace{1em}Other & 0.0\% & 0.0\% & 0.0\% & 0.8\%\\
\hspace{1em}White & 30.1\% & 22.3\% & 26.8\% & 56.1\%\\
\hspace{1em}Unknown & 5.0\% & 4.5\% & 4.8\% & 6.9\%\\
\addlinespace[0.3em]
\multicolumn{5}{l}{\textbf{Gender}}\\
\hspace{1em}Female & 58.6\% & 53.5\% & 56.3\% & 52.4\%\\
\hspace{1em}Male & 40.6\% & 45.7\% & 42.9\% & 47.2\%\\
\hspace{1em}Multiple & 0.8\% & 0.8\% & 0.8\% & 0.4\%\\
\bottomrule
\multicolumn{5}{l}{\textit{Note: }}\\
\multicolumn{5}{l}{Percentages may not add up to 100 because of missing data}\\
\end{tabu}
\end{table}
\section{Public Housing and Diabetes}\label{public-housing-and-diabetes}

For the primary analysis, the association between diabetes status and
public housing assistance, the crude model showed that the odds ratio of
having diabetes was 1.34 fold greater for those receiving public housing
assistance (See Table \ref{tab:table2}). This effect increased in the
adjusted model, PHA residents were 94\% more likely to meet the
definition of diabetes compared to those that were non-PHA residents
(See Table \ref{tab:table3}).
\begin{longtable}[]{@{}lll@{}}
\caption{\label{tab:PHA} Association between PHA Status and
Diabetes}\tabularnewline
\toprule
Housing Status & Model 1 & Model 2\tabularnewline
\midrule
\endfirsthead
\toprule
Housing Status & Model 1 & Model 2\tabularnewline
\midrule
\endhead
Non-PHA & Referent & Referent\tabularnewline
PHA & 1.34 (CI: 1.31-1.38) & 1.94 (CI: 1.88-1.99)\tabularnewline
\bottomrule
\end{longtable}
\begin{longtable}[]{@{}lll@{}}
\caption{\label{tab:PHAA} Association between the Public Housing Authorities
and Diabetes}\tabularnewline
\toprule
Status & Model 1 & Model 2\tabularnewline
\midrule
\endfirsthead
\toprule
Status & Model 1 & Model 2\tabularnewline
\midrule
\endhead
Non-PHA & Referent & Referent\tabularnewline
KCHA & 1.28 (CI: 1.24-1.33) & 2.16 (CI: 2.09-2.25)\tabularnewline
SHA & 1.42 (CI: 1.38-1.48) & 1.70 (CI: 1.64-1.77)\tabularnewline
\bottomrule
\end{longtable}
\section{Public Housing Authorities and
Diabetes}\label{public-housing-authorities-and-diabetes}

In the second analysis, measuring the association between diabetes
status and the specific public housing authorities, the crude model
showed that the odds of meeting the definition of diabetes were 1.28
times greater among KCHA residents and 1.42 times greater among SHA
residents (See Table \ref{tab:table4}). The adjusted model revealed that
among KCHA residents, the odds of meeting the definition of diabetes
were 2.16 times higher and 1.70 for SHA residents compared to non-PHA
residents (See Table \ref{tab:table5}).

\chapter{Discussion}\label{discussion}

\section{Discussion}\label{discussion-1}

Findings from this study indicate that public housing assistance was
positively associated with diabetes status. After adjusting for
potential confounders (age, gender, race and ethnicity) the effect of
public housing on diabetes status increased even greater. Furthermore,
findings also suggest that when stratified into PHA agency in the crude
model, KCHA residents where were less likely to meet the definition of
diabetes than SHA residents but still more likely than the non-PHA
group. After adjust for potential confounders, the effect of association
between PHA agency on diabetes saw a greater increase for KCHA residents
than SHA residents.

The increased risk of diabetes observed in this study in relation to
public housing had similar results to another study that compared public
housing residents in Boston, MA to other city residents for health
outcomes including diabetes that revealed worse health outcomes for
public housing residents (Digenis-Bury et al., 2008). A potential
explanation of these findings may be attributed to the fact that public
housing residents were more likely to be racial and ethnic minorities
and the prevalence of diabetes is often greater for racial and ethnic
populations (Chow, Foster, Gonzalez, \& McIver, 2012). Another possible
explanation is that areas and neighborhoods with a high level of
poverty, such as where public housing properties may be located tend to
also have a higher prevalence of obesity and diabetes (Ludwig et al.,
2011).

These findings suggest that public housing residents have a greater need
for diabetes treatment and could be an avenue in which resources on
diabetes prevention and management may be more effective. A health
promotion intervention that utilized homecare nurses to implement a
diabetes prevention program in public housing communities proved to be
successful and can be a viable option in improving health outcomes
(Whittemore, Rosenberg, Gilmore, Withey, \& Breault, 2014). Considering
the population demographics in the PHA group, another successful
strategy may be to implement a culturally competent diabetes care
intervention program (Zeh, Sandhu, Cannaby, \& Sturt, 2012).

In contrast, the preliminary findings in this study that suggests that
PHA recipients are more likely to be diabetic are inferential and
therefore cannot be interpreted as a causal relationship. As previously
mentioned, substandard and unstable housing has been linked with worse
diabetes health outcomes, however stable housing provided by PHA has
been associated with the ability for participants to afford
diabetes-related financial expenses (Keene, Henry, Gormley, \& Ndumele,
2018). Additionally, when housing needs are met participants are able to
prioritize diabetes self-management, avoiding potential complications
(Keene et al., 2018). Those participants with diabetes prior to
receiving public housing assistance could potentially see a reduction in
diabetes related complications over time. Strengths of this study
include the population-based study design that allows for estimation of
disease impact on a broad scale and the well-defined study population.
The data used in this study were also from reliable sources with careful
and normally complete documentation, subsequently reducing the impact of
information bias.

\section{Limitations}\label{limitations}

There are several limitations to note. First, there may have been
unmeasured potential confounders. Given the data provided, other
elements that are recognized to be potential confounders for diabetes
may have been useful to include in this study like socioeconomic status
and other health characteristics like BMI.

Another limitation is that this study does not provide the prevalence of
diabetes due to the inherent characteristic of claims data. The
population captured in the study were only those that sought healthcare
services for diabetes related outcomes. People who may have been
diabetic during this period but were asymptomatic or those who had been
previously diagnosed with diabetes but did not seek care within the time
frame provided by the CCW algorithm definition of diabetes were not
captured in the study. Consequently, the eligible population in this
study cannot provide a prevalence estimate by themselves.

Despite the limitations, this study contributes to our understanding of
poverty and diabetes self-management the findings are generalizable to
low-income, racially, and ethnically diverse populations with diabetes
who obtain health care in safety-net health settings. Future studies
should continue investigating the relationship between public housing
and diabetes and in particular further studies should explore the
association over a longer time period.

\appendix

\chapter*{Appendix}\label{appendix}
\addcontentsline{toc}{chapter}{Appendix}
\begin{table}

\caption{\label{tab:table2}Crude PHA Regression Model}
\centering
\fontsize{12}{14}\selectfont
\begin{tabu} to \linewidth {>{\raggedright}X>{\raggedleft}X>{\raggedleft}X>{\raggedright}X>{\raggedleft}X>{\raggedleft}X}
\toprule
Term & Odds Ratio & SE & P-Value & 95\% CI Low & 95\% CI High\\
\midrule
(Intercept) & 0.11 & 0.00 & 0 & 0.11 & 0.11\\
pha & 1.35 & 0.01 & <0.05 & 1.31 & 1.38\\
\bottomrule
\end{tabu}
\end{table}
\begin{table}

\caption{\label{tab:table3}Adjusted PHA Regression Model}
\centering
\fontsize{12}{14}\selectfont
\begin{tabu} to \linewidth {>{\raggedright}X>{\raggedleft}X>{\raggedleft}X>{\raggedright}X>{\raggedleft}X>{\raggedleft}X}
\toprule
Term & Odds Ratio & SE & P-Value & 95\% CI Low & 95\% CI High\\
\midrule
(Intercept) & 0.02 & 0.05 & <0.05 & 0.01 & 0.02\\
pha & 1.94 & 0.01 & <0.05 & 1.88 & 1.99\\
Age & 1.04 & 0.00 & <0.05 & 1.04 & 1.04\\
Male & 1.17 & 0.01 & <0.05 & 1.15 & 1.19\\
Multiple & 1.33 & 0.08 & <0.05 & 1.13 & 1.56\\
\addlinespace
Asian & 1.02 & 0.05 & 0.71 & 0.93 & 1.12\\
Asian PI & 0.37 & 0.05 & <0.05 & 0.33 & 0.41\\
Black & 0.82 & 0.05 & <0.05 & 0.75 & 0.90\\
Latino & 0.64 & 0.05 & <0.05 & 0.58 & 0.71\\
Multiple & 0.81 & 0.05 & <0.05 & 0.74 & 0.89\\
\addlinespace
NH/PI & 1.54 & 0.05 & <0.05 & 1.39 & 1.71\\
Other & 0.42 & 0.06 & <0.05 & 0.37 & 0.47\\
Unknown & 0.61 & 0.05 & <0.05 & 0.56 & 0.68\\
White & 0.36 & 0.05 & <0.05 & 0.33 & 0.40\\
\bottomrule
\end{tabu}
\end{table}
\begin{table}

\caption{\label{tab:table4}Crude PHA Agency Regression Model}
\centering
\fontsize{12}{14}\selectfont
\begin{tabu} to \linewidth {>{\raggedright}X>{\raggedleft}X>{\raggedleft}X>{\raggedright}X>{\raggedleft}X>{\raggedleft}X}
\toprule
Term & Odds Ratio & SE & P-Value & 95\% CI Low & 95\% CI High\\
\midrule
(Intercept) & 0.11 & 0.00 & <0.05 & 0.11 & 0.11\\
KCHA & 1.29 & 0.02 & <0.05 & 1.25 & 1.33\\
SHA & 1.43 & 0.02 & <0.05 & 1.38 & 1.48\\
\bottomrule
\end{tabu}
\end{table}
\begin{table}

\caption{\label{tab:table5}Adjusted PHA Agency Regression Model}
\centering
\fontsize{12}{14}\selectfont
\begin{tabu} to \linewidth {>{\raggedright}X>{\raggedleft}X>{\raggedleft}X>{\raggedright}X>{\raggedleft}X>{\raggedleft}X}
\toprule
Term & Odds Ratio & SE & P-Value & 95\% CI Low & 95\% CI High\\
\midrule
(Intercept) & 0.02 & 0.05 & <0.05 & 0.01 & 0.02\\
KCHA & 2.17 & 0.02 & <0.05 & 2.09 & 2.25\\
SHA & 1.71 & 0.02 & <0.05 & 1.64 & 1.78\\
Age & 1.04 & 0.00 & <0.05 & 1.04 & 1.04\\
Male & 1.17 & 0.01 & <0.05 & 1.15 & 1.19\\
\addlinespace
Multiple & 1.34 & 0.08 & <0.05 & 1.14 & 1.57\\
Asian & 1.02 & 0.05 & 0.72 & 0.93 & 1.12\\
Asian PI & 0.37 & 0.05 & <0.05 & 0.33 & 0.40\\
Black & 0.82 & 0.05 & <0.05 & 0.75 & 0.90\\
Latino & 0.64 & 0.05 & <0.05 & 0.58 & 0.71\\
\addlinespace
Multiple & 0.80 & 0.05 & <0.05 & 0.73 & 0.88\\
NH/PI & 1.53 & 0.05 & <0.05 & 1.38 & 1.70\\
Other & 0.41 & 0.06 & <0.05 & 0.37 & 0.47\\
Unknown & 0.61 & 0.05 & <0.05 & 0.55 & 0.67\\
White & 0.36 & 0.05 & <0.05 & 0.33 & 0.39\\
\bottomrule
\end{tabu}
\end{table}
\backmatter

\chapter*{References}\label{references}
\addcontentsline{toc}{chapter}{References}

\markboth{References}{References}

\noindent

\setlength{\parindent}{-0.20in} \setlength{\leftskip}{0.20in}
\setlength{\parskip}{8pt}

\hypertarget{refs}{}
\hypertarget{ref-Bullard2018}{}
Bullard, K. M., Cowie, C. C., Lessem, S. E., Saydah, S. H., Menke, A.,
Geiss, L. S., \ldots{} Imperatore, G. (2018). Prevalence of Diagnosed
Diabetes in Adults by Diabetes Type --- United States, 2016. \emph{MMWR.
Morbidity and Mortality Weekly Report}, \emph{67}(12), 359--361.
\url{http://doi.org/10.15585/mmwr.mm6712a2}

\hypertarget{ref-CDC2020}{}
CDC. (2020). \emph{National Diabetes Statistics Report 2020. Estimates
of diabetes and its burden in the United States.}

\hypertarget{ref-CMS2020}{}
Centers for Medicare and Medicaid Services. (2020). Condition Categories
- Chronic Conditions Data Warehouse. Retrieved from
\url{https://www2.ccwdata.org/web/guest/condition-categories}

\hypertarget{ref-Chow2012}{}
Chow, E. A., Foster, H., Gonzalez, V., \& McIver, L. S. (2012). The
disparate impact of diabetes on racial/ethnic minority populations.
\emph{Clinical Diabetes}, \emph{30}(3), 130--133.
\url{http://doi.org/10.2337/diaclin.30.3.130}

\hypertarget{ref-Clark2014}{}
Clark, M. L. (2014). Social determinants of type 2 diabetes and health
in the United States. \emph{World Journal of Diabetes}, \emph{5}(3),
296. \url{http://doi.org/10.4239/wjd.v5.i3.296}

\hypertarget{ref-Digenis-Bury2008}{}
Digenis-Bury, E. C., Brooks, D. R., Chen, L., Ostrem, M., \& Horsburgh,
C. R. (2008). Use of a population-based survey to describe the health of
boston public housing residents. \emph{American Journal of Public
Health}, \emph{98}(1), 85--91.
\url{http://doi.org/10.2105/AJPH.2006.094912}

\hypertarget{ref-Divers2020}{}
Divers, J., Mayer-Davis, E. J., Lawrence, J. M., Isom, S., Dabelea, D.,
Dolan, L., \ldots{} Wagenknecht, L. E. (2020). Trends in Incidence of
Type 1 and Type 2 Diabetes Among Youths --- Selected Counties and Indian
Reservations, United States, 2002--2015. \emph{MMWR. Morbidity and
Mortality Weekly Report}, \emph{69}(6), 161--165.
\url{http://doi.org/10.15585/mmwr.mm6906a3}

\hypertarget{ref-Greenland1988}{}
Greenland, S., \& Morgenstern, H. (1988). Classification schemes for
epidemiologic research designs. \emph{Journal of Clinical Epidemiology},
\emph{41}(8), 715--716.
\url{http://doi.org/10.1016/0895-4356(88)90155-2}

\hypertarget{ref-Hayes-Larson2019}{}
Hayes-Larson, E., Kezios, K. L., Mooney, S. J., \& Lovasi, G. (2019).
Who is in this study, anyway? Guidelines for a useful Table 1.
\emph{Journal of Clinical Epidemiology}, \emph{114}, 125--132.
\url{http://doi.org/10.1016/j.jclinepi.2019.06.011}

\hypertarget{ref-HUD2020}{}
HUD. (2020). U.S. Department of Housing and Urban Development (HUD).
Retrieved from \url{https://www.hud.gov/}

\hypertarget{ref-Kautzky-Willer2016}{}
Kautzky-Willer, A., Harreiter, J., \& Pacini, G. (2016). Sex and gender
differences in risk, pathophysiology and complications of type 2
diabetes mellitus. \emph{Endocrine Reviews}, \emph{37}(3), 278--316.
\url{http://doi.org/10.1210/er.2015-1137}

\hypertarget{ref-Keene2018}{}
Keene, D. E., Henry, M., Gormley, C., \& Ndumele, C. (2018). 'Then I
Found Housing and Everything Changed': Transitions to Rent-Assisted
Housing and Diabetes Self-Management. \emph{Cityscape (Washington,
D.C.)}, \emph{20}(2), 107--118. \url{http://doi.org/10.2307/26472170}

\hypertarget{ref-Krieger2002}{}
Krieger, J., \& Higgins, D. L. (2002). Housing and health: Time again
for public health action. \emph{American Journal of Public Health},
\emph{92}(5), 758--768. \url{http://doi.org/10.2105/AJPH.92.5.758}

\hypertarget{ref-Ludwig2011}{}
Ludwig, J., Sanbonmatsu, L., Gennetian, L., Adam, E., Duncan, G. J.,
Katz, L. F., \ldots{} McDade, T. W. (2011). Neighborhoods, obesity, and
diabetes - A randomized social experiment. \emph{New England Journal of
Medicine}, \emph{365}(16), 1509--1519.
\url{http://doi.org/10.1056/NEJMsa1103216}

\hypertarget{ref-Manjarrez2007}{}
Manjarrez, C. A., Popkin, S. J., \& Guernsey, E. (2007). \emph{Poor
Health: Adding Insult to Injury for HOPE VI Families}. Urban Institute.

\hypertarget{ref-PHSKC2018}{}
Public Health - Seattle \& King County. (2018). \emph{King County Data
Across Sectors for Housing and Health, 2018}. Public Health - Seattle \&
King County.

\hypertarget{ref-Selvin2013}{}
Selvin, E., \& Parrinello, C. M. (2013). Age-related differences in
glycaemic control in diabetes. \emph{Diabetologia}, \emph{56}(12),
2549--2551. \url{http://doi.org/10.1007/s00125-013-3078-7}

\hypertarget{ref-Sperandei2014}{}
Sperandei, S. (2014). Understanding logistic regression analysis.
\emph{Biochemia Medica}, \emph{24}(1), 12--18.
\url{http://doi.org/10.11613/BM.2014.003}

\hypertarget{ref-Taylor2018}{}
Taylor, L. (2018). ``Housing And Health: An Overview Of The Literature,
" Health Affairs Health Policy Brief. \emph{Health Affairs}.
\url{http://doi.org/10.1377/hpb20180313.396577}

\hypertarget{ref-Thomson2013}{}
Thomson, H., Thomas, S., Sellstrom, E., \& Petticrew, M. (2013). Housing
improvements for health and associated socio-economic outcomes.
\emph{Cochrane Database of Systematic Reviews}, \emph{2013}(2).
\url{http://doi.org/10.1002/14651858.CD008657.pub2}

\hypertarget{ref-Vijayaraghavan2011}{}
Vijayaraghavan, M., Jacobs, E. A., Seligman, H., \& Fernandez, A.
(2011). The Association Between Housing Instability, Food Insecurity,
and Diabetes Self-Efficacy in Low-Income Adults - PubMed. \emph{Journal
of Health Care for the Poor and Underserved}, \emph{22}(4). Retrieved
from \url{https://pubmed.ncbi.nlm.nih.gov/22080709/}

\hypertarget{ref-Whittemore2014}{}
Whittemore, R., Rosenberg, A., Gilmore, L., Withey, M., \& Breault, A.
(2014). Implementation of a Diabetes Prevention Program in Public
Housing Communities. \emph{Public Health Nursing}, \emph{31}(4),
317--326. \url{http://doi.org/10.1111/phn.12093}

\hypertarget{ref-Zeh2012}{}
Zeh, P., Sandhu, H. K., Cannaby, A. M., \& Sturt, J. A. (2012). The
impact of culturally competent diabetes care interventions for improving
diabetes-related outcomes in ethnic minority groups: A systematic
review. \emph{Diabetic Medicine}, \emph{29}(10), 1237--1252.
\url{http://doi.org/10.1111/j.1464-5491.2012.03701.x}
\end{document}
